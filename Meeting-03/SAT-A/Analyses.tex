\documentclass{article}

% Packages
\usepackage[T1]{fontenc}
\usepackage[scaled]{helvet}

\usepackage[top=1in,bottom=1in,left=1in,right=1in]{geometry}
\usepackage{needspace}
\usepackage{tikz}
\usepackage{graphicx}
\usepackage{hyperref}
\usepackage{xcolor}
\usepackage{array}
\usepackage{booktabs}

\graphicspath{{./img/}}
\hypersetup{
  colorlinks,
  linkcolor={red!50!black},
  citecolor={blue!50!black},
  urlcolor={blue!80!black}
}

% Metadata
\author{Roman Shchekotov}
\title{SAT Analyses}
\date{\today}

% Generate '@...'-commands for later use
\makeatletter

% Environment for the SAT Analysis
\newenvironment{SATAnalysis}{
  \vspace{0.25cm}
  \begin{tabular}{>{\raggedright\arraybackslash}p{0.45\textwidth} >{\raggedleft\arraybackslash}p{0.45\textwidth}}
  {\Large \textsc{Upstream Causes}} & {\Large \textsc{Downstream Effects}}\\
}{
  \bottomrule
  \end{tabular}
  \vspace{0.5cm}
}
\newcommand{\SATStructural}{\midrule\multicolumn{2}{c}{\Large \textsc{Structural}}\\ \midrule}
\newcommand{\SATAttitudinal}{\midrule\multicolumn{2}{c}{\Large \textsc{Attitudinal}}\\ \midrule}
\newcommand{\SATTransactional}{\midrule\multicolumn{2}{c}{\Large \textsc{Transactional}}\\ \midrule}
\newcommand{\SATCause}[1]{#1 & \\}
\newcommand{\SATEffect}[1]{ & #1\\}
\newcommand{\SATCauseEffect}[2]{#1 & #2\\}

% 2x Newlines
\newcommand{\nn}{\vspace{\baselineskip}\\}
% Header Function
\newcommand{\showtitle}{\begin{center}\textbf{\Huge\textsc{\@title}}\end{center}}
% Highlight Enabling / Inhibiting Theme
\newcommand{\hletheme}[1]{\textbf{\color{teal}{#1}}}
%\newcommand{\hlletheme}[2]{\textbf{\color{teal}\href{#2}{#1}}}
\newcommand{\hlitheme}[1]{\textbf{\color{purple}{#1}}}

% External Link Graphic
\newcommand{\ExternalLink}{%
  \tikz[x=1.2ex, y=1.2ex, baseline=-0.05ex]{% 
    \begin{scope}[x=1ex, y=1ex]
      \clip (-0.1,-0.1) 
        --++ (-0, 1.2) 
        --++ (0.6, 0) 
        --++ (0, -0.6) 
        --++ (0.6, 0) 
        --++ (0, -1);
      \path[draw, 
        line width = 0.5, 
        rounded corners=0.5] 
        (0,0) rectangle (1,1);
    \end{scope}
    \path[draw, line width = 0.5] (0.5, 0.5) -- (1, 1);
    \path[draw, line width = 0.5] (0.6, 1) -- (1, 1) -- (1, 0.6);
  }
}
% External Link Macro
\newcommand{\ehref}[2]{\href{#1}{#2 \ExternalLink}}

\begin{document}
\showtitle
\begin{figure}[ht!]
  \centering
  \includegraphics[width=0.5\textwidth]{Recycle}
\end{figure}
\tableofcontents

\newpage

\section{Theme Collection}
\subsection{Enablers}
The following themes are considered to be ``enablers'' for the topic of
``Ecological waste management in the Philippines''.

\Needspace{8\baselineskip}
\subsubsection{Care of the environment}
\label{subsubsec:care-of-the-environment}
The topic `care of the environment' is of utmost
\hletheme{importance to the youth and young adults}, as that directly affects
their future lifestyle and that of future generations.
Especially the youth, a global example of which would be 
\ehref{https://en.wikipedia.org/wiki/Greta_Thunberg}{Greta Thunberg},
is rather passionate and active in the topic of environmental protection.
\nn
This care for the environment is not only limited to the youth.
These movements \hletheme{attract volunteers from all corners of the world}
and lead to a collaborative effort to protect the environment with
knowledge, experiences and perspectives from all over the world. 
Especially from the places where some of the issues faced in the 
Philippines are not as prevalent and solutions are known and implemented.
\nn
Aside from helping hands coming from outside the Philippines, there
also are \hletheme{local initiatives with insight on the consequences} of
the issues.
Not only do the participants of such have the aspiration to solve the
problems, such as the other groups mentioned above, but they also have
to deal with the consequences of the issues on a daily basis.

\Needspace{8\baselineskip}
\subsubsection{Low waste mentality}
\label{subsubsec:low-waste-mentality}
There already exist plenty of \hletheme{low waste concepts \& practices}
that are being implemented worldwide and also exist to some extent in
the Philippines.
In particular concepts such as the
\ehref{https://en.wikipedia.org/wiki/Waste_hierarchy}{Waste Hierarchy}
are pretty well known and contribute to some of the low waste mentality.
\nn
Naturally, there is also a business aspect that was born from the
low waste mentality.
Entrepeneurs started to create \hletheme{sustainable products} that
either reduce the waste produced (i.e. avoiding packaging where possible)
or make an existing product more sustainable (i.e. reusable coffee cups).
\nn
Such \hletheme{innovations lead by the ``environmentalists''} are
what brings new concepts and awareness to the general public.
Through conventions and other events, the mentality spreads and
more people become aware of the issue.

\Needspace{8\baselineskip}
\subsubsection{Reinforcement of a sustainable lifestyle through legislation}
\label{subsubsec:reinforcement-of-a-sustainable-lifestyle-through-legislation}
In order to face the issues the Republic of the Philippines is facing,
several legislations have been passed that aim to solve the problems.
An example of such is the
\ehref{https://web.archive.org/web/20221212111636/https://www.officialgazette.gov.ph/2001/01/26/republic-act-no-9003-s-2001/}{\hletheme{Republic Act No. 9003}},
also known as the ``Ecological Solid Waste Management Act of 2000''.
It defines the procedures and institutions necessary to mitigate the
problems of waste management in the Philippines.
\nn
Another instance of legislation that aims to solve the issues is the
\ehref{https://web.archive.org/web/20230213214528/https://www.officialgazette.gov.ph/2022/04/15/republic-act-no-11698/}{\hletheme{Republic Act No. 11698}}, 
which amends the previous legislation and aims to further improve 
the waste management in the Philippines by introducing the ``polluter pays'' 
principle.
These acts are also partially a result of international pressure, such
as the UN and in particular their 11th Sustainable Development Goal
\footnote{\ehref{https://www.un.org/sustainabledevelopment/cities/}{UN Sustainable Development Goal 11}}.
\nn
Lastly, another highly important step towards a sustainable lifestyle is
a functioning \hletheme{Garbage Collection \& Recycling System}, which
is also a government responsibility and is therefore also covered by
the two Acts mentioned above.


\Needspace{8\baselineskip}
\subsubsection{Access to information}
\label{subsubsec:access-to-information}
A huge step towards the desired outcome is also achieved through the
\hletheme{globally available access to information}. We live in a time
where information is readily available to the vast majority of the world,
which makes finding solutions to problems significantly easier.
\nn
This information becomes even more impactful when taught by people who
have experience with the respective topic and can \hletheme{share the 
knowledge through education}.
That way the information is not only available, but also understood and
can be applied in the given situation.

\subsection{Inhibitors}
The following themes are considered to be ``inhibitors'' for the topic of
``Ecological waste management in the Philippines''.

\Needspace{8\baselineskip}
\subsubsection{Lack of executive action backing up legislation}
\label{subsubsec:lack-of-executive-action-backing-up-legislation}
Although there are several legislations that aim to solve the issues
faced by the Republic of the Philippines, there is a \hlitheme{lack 
of government action} to enforce the legislation.
\nn
This can be seen in various ways, such as the \hlitheme{need for more
designated trash bins}.
Separating the trash into different categories is a crucial step in
the waste management process, which is why it is important to have
trash bins that would allow at least the separation of bio-degradable,
recyclable and non-recyclable waste.
\nn
Another example of a lack of government action is the \hlitheme{poor
garbage collection schedule and system}.
Distant towns and villages are often not visited by the garbage trucks
and therefore have to wait for a longer period of time until their
waste is collected.
In such situations the waste is often left on the streets or
illegal dumping sites, which is counterproductive to the achievement
of our goal.

\Needspace{8\baselineskip}
\subsubsection{Lack of awareness \& education}
\label{subsubsec:lack-of-awareness-education}
Thanks to the internet the access to information is easier than ever,
and yet a lot of people are not aware of the consequences of simple
things, such as littering.
People generally do not have a reason to look up information about
the topic of waste management, thus there is a \hlitheme{need in public
information education campaigns}.
\nn
Corelating to the lack of awareness is also the \hlitheme{insufficience
of insight on the relationships: environment $\Leftrightarrow$ human
health and survival}.
The environment is a crucial part of our lives and yet we often do not
realize how much we depend on it.
It's pollution and degradation can have a huge impact on our health
starting from making it impossible to breathe fresh air,
such as is seen in some industrial parts of China
\footnote{\ehref{https://www.who.int/china/health-topics/air-pollution}{WHO: Air pollution in China}},
to poisoning food and water sources
\footnote{i.e.: \ehref{https://www.who.int/news-room/fact-sheets/detail/lead-poisoning-and-health}{WHO: Lead Poisoning}}.
\nn
There also is the aspect of \hlitheme{``no knowledge of other life''}.
Since there never has been a proper system of waste management in the
Philippines, people are not used to the idea of recycling and reusing
materials.
Moreover, why would you need to change your attitude towards waste
management when it hasn't been an issue in the past?
The change of habits is a difficult process and requires at least some
motivation from each individual to succeed, which has yet to be achieved.

\Needspace{8\baselineskip}
\subsubsection{Individualistic mentality}
\label{subsubsec:individualistic-mentality}
The mentality of each individual is a crucial factor in the success of
the broad and sustained change for the sake of the environment.
Often a person has \hlitheme{too much distance to the problem} to
realize the consequences of their actions and be farsighted enough to
understand the lasting impact, especially when it comes to problems
with such a large scale and slow, but consistent development as the 
degradation of the environment.
\nn
Add to that another aspect of the ``individualistic mentality'':
\hlitheme{``one person can't make a change''}.
This is a common thought that is often used as an excuse to not
change one's behavior, as one's change would not be enough to make a
difference on a scale that matters.

\subsubsection{Limited access to sustainable products}
\label{subsubsec:limited-access-to-sustainable-products}
When it comes to sustainable products produce through innovative
ideas and technologies brought forward by environmental-friendly
companies, the \hlitheme{limited access to such products} is a huge
inhibitor to the positive effect it would have on the environment.
Such products are often more expensive than their non-sustainable
counterparts, which makes them inaccessible to the vast majority
of the population.
Sometimes the issue lies not even in the cost of the product, but
the local availability instead, as the trend of sustainable products
is rather new and therefore not yet widespread.

\newpage

\section{Analyses}
\subsection{\nameref{subsubsec:care-of-the-environment}}
\begin{SATAnalysis}
  \SATStructural
  \SATCauseEffect{Young Generation}{Personal Education [i.e. Talks]}
  \SATCauseEffect{Local Community Influence}{Environment NGOs \& Volunteers}
  \SATCauseEffect{International Collaboration}{Educated and Proactive Youth}
  \SATAttitudinal
  \SATCauseEffect{Appreciation of the environment}{Support for NGOs}
  \SATCauseEffect{Desire for a better future}{Collaborative Env. Contributions}
  \SATTransactional
  \SATCauseEffect{Community Leaders / Elders}{Environment-conscious leaders}
  \SATCauseEffect{Environment Influencers}{Environment Influencers}
  \SATCause{Family influence (childhood upbringing)}
  \SATCause{Environment Activists}
\end{SATAnalysis}

\subsection{\nameref{subsubsec:low-waste-mentality}}
\begin{SATAnalysis}
  \SATStructural
  \SATCauseEffect{Known Problem \& Awareness of Solutions}{Environmentally Conscious Businesses}
  \SATCauseEffect{Support with diverse perspectives}{Environmentally Conscious Individuals}
  \SATEffect{Litterers}
  \SATAttitudinal
  \SATCauseEffect{Ethical Motivation to Businesses}{Out of the Box Thinking}
  \SATCauseEffect{Adaptability of Individuals}{Eco-Attitude Shift}
  \SATTransactional
  \SATCauseEffect{Environmental Groups}{Non-Profit Organizations}
  \SATCauseEffect{Forward-thinking Businesses}{Non Government Organizations}
  \SATCauseEffect{Foreign Volunteers}{Environmental Activists}
\end{SATAnalysis}

\subsection{\nameref{subsubsec:reinforcement-of-a-sustainable-lifestyle-through-legislation}}
\begin{SATAnalysis}
  \SATStructural
  \SATCauseEffect{Communities \& Towns}{Law Enactment}
  \SATCauseEffect{Budget Appropriation}{Waste Man. Infrastructure Expansion}
  \SATCauseEffect{Law Enactment}{Decrease in Polluting Businesses}
  \SATAttitudinal
  \SATCauseEffect{Bureacratic Approach}{Supportive Entities (implementing change)}
  \SATCauseEffect{International Pressure}{Defiant Entities (resisting change)}
  \SATEffect{Neutral Entities (no inconvenience by change)}
  \SATEffect{Openness to Partnerships}
  \SATTransactional
  \SATCauseEffect{Government}{Supportive Communities}
  \SATCauseEffect{Local Governers (Town Mayors, etc.)}{Defiant Communities}
  \SATCause{Unified Community Voice}
  \SATCause{United Nations}
\end{SATAnalysis}

\subsection{\nameref{subsubsec:access-to-information}}
\begin{SATAnalysis}
  \SATStructural
  \SATCauseEffect{Internet Access}{Education on the Web}
  \SATCauseEffect{On-/Offline Education}{Educated Youth}
  \SATEffect{Broader Availability of products}
  \SATEffect{International Collaboration}
  \SATAttitudinal
  \SATCauseEffect{Curiosity}{Knowledge Exchange (Lakbay Aral)}
  \SATCauseEffect{Receptiveness}{Improved Problem Solving Abilities}
  \SATEffect{Broader Perspective \& Understanding}
  \SATTransactional
  \SATCauseEffect{Educational Institutions}{Educated Specialists}
  \SATCause{MOOC Professors \& TA's}
\end{SATAnalysis}

\subsection{\nameref{subsubsec:lack-of-executive-action-backing-up-legislation}}
\begin{SATAnalysis}
  \SATStructural
  \SATCauseEffect{Lack of Designated Trash Bins}{Disorganized Waste Management}
  \SATCauseEffect{Poor Garbage Collection Schedule \& System}{Formation of Illegal Dumpsites}
  \SATCauseEffect{Missing Recycling Infrastructure}{Poisoning of the Env. with Toxic Waste}
  \SATEffect{Impossibility for Recycling (missing segregation)}
  \SATAttitudinal
  \SATCauseEffect{Prolonged Negligence}{Careless Behavior with Waste}
  \SATEffect{Provisional Makeshift Solutions}
  \SATTransactional
  \SATCauseEffect{Tentative Actions by the Government}{Neglected Remote Areas}
  \SATEffect{Litterers}
\end{SATAnalysis}

\subsection{\nameref{subsubsec:lack-of-awareness-education}}
\begin{SATAnalysis}
  \SATStructural
  \SATCauseEffect{Lack of public IEC}{Lack of awareness}
  \SATCauseEffect{Lack of awareness}{Lack of education}
  \SATCauseEffect{Lack of insight into Human $\Leftrightarrow$ Nature relations}{}
  \SATAttitudinal
  \SATCauseEffect{No knowledge of other life}{Careless Attitude}
  \SATCauseEffect{``If it's not broken, don't fix it''}{Ignorance}
  \SATTransactional
  \SATCauseEffect{Each Individual}{Careless Individuals}
\end{SATAnalysis}

\subsection{\nameref{subsubsec:individualistic-mentality}}
\begin{SATAnalysis}
  \SATStructural
  \SATCauseEffect{Overwhelming Scale of the Problem}{Consistent Degradation of the Environment}
  \SATAttitudinal
  \SATCauseEffect{Distance to the Issue}{Ignorance of Long-Term Effects}
  \SATCauseEffect{``One person can't make a difference''}{Preference of Short-Term Solutions}
  \SATEffect{Individualistic World View}
  \SATTransactional
  \SATCauseEffect{Each Individual}{Overwhelmed Citizens}
  \SATEffect{Ignorant Citizens}
\end{SATAnalysis}

\subsection{\nameref{subsubsec:limited-access-to-sustainable-products}}
\begin{SATAnalysis}
  \SATStructural
  \SATCauseEffect{Limited Supply}{Limited Access}
  \SATCauseEffect{Costly Products}{Low Affordability}
  \SATEffect{Low Profit Margin for Businesses}
  \SATAttitudinal
  \SATCauseEffect{Consumers lacking awareness of alternatives}{Fallback to Non-Sustainable Products}
  \SATTransactional
  \SATCauseEffect{Each Individual}{Lack of Support for Sustainable Businesses}
  \SATCauseEffect{Eco-friendly Businesses}{}
\end{SATAnalysis}

\end{document}